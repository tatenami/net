% ファイル先頭から\begin{document}までの内容(プレアンブル)については,
% 基本的に { } の中を書き換えるだけでよい.
\documentclass[autodetect-engine,dvi=dvipdfmx,ja=standard,
               a4j,11pt]{bxjsarticle}

%%======== プレアンブル ============================================%%
% 用紙設定:指示があれば,適切な余白に設定しなおす
\RequirePackage{geometry}
\geometry{reset,paperwidth=210truemm,paperheight=297truemm}
\geometry{hmargin=25truemm,top=20truemm,bottom=25truemm,footskip=10truemm,headheight=0mm}
%\geometry{showframe} % 本文の"枠"を確認したければ,コメントアウト

% 設定:図の挿入
% http://www.edu.cs.okayama-u.ac.jp/info/tool_guide/tex.html#graphicx
\usepackage{graphicx}

% 設定:ソースコードの挿入
% http://www.edu.cs.okayama-u.ac.jp/info/tool_guide/tex.html#fancyvrb
\usepackage{fancyvrb}
\renewcommand{\theFancyVerbLine}{\texttt{\footnotesize{\arabic{FancyVerbLine}:}}}

%%======== レポートタイトル等 ======================================%%
% ToDo: 提出要領に従って,適切なタイトル・サブタイトルを設定する
\title{情報工学実験C ネットワーク実験課題レポート \\
       |\large{クライアントサーバーモデルで動作する名簿管理プログラムの作成}|}

% ToDo: 自分自身の氏名と学生番号に書き換える
\author{氏名: 寺岡 久騎 (TERAOKA, Hisaki) \\
        学生番号: 09B22433}

% ToDo: レポート課題等の指示に従って適切に書き換える
\date{出題日: 2024年12月13日 \\
      提出日: 2025年 1月21日 \\
      締切日: 2025年 1月21日 \\}  % 注:最後の\\は不要に見えるが必要.


%%======== 本文 ====================================================%%
\begin{document}
\maketitle
% 目次つきの表紙ページにする場合はコメントを外す
%{\footnotesize \tableofcontents \newpage}

%% 本文は以下に書く.課題に応じて適切な章立てを構成すること.
%% 章=\section,節=\subsection,項=\subsubsection である.

%--------------------------------------------------------------------%
% \section{概要} \label{sec:abstract}


%--------------------------------------------------------------------%
\section{プログラムの処理の概要および作成方針}
本実験では,ネットワーク通信により動作するクライアントサーバーモデルの
TCPによる相互のデータ送受信を行う名簿管理プログラムの作成を行った.
本章ではサーバ・クライアントプログラムのそれぞれの機能と,
プログラム全体としての処理の概要について解説する.

\subsection{プログラム全体の処理の流れと概要} \label{sec:prog-flow}
プログラム全体の処理の流れとして,まず
サーバはクライアントからのTCPの接続要求を待ち受け,クライアント
はサーバに接続する.
コネクションを確立することで名簿管理の処理が開始し,以降はシステムコール関数を用いたソケット通信により
以下の一連の処理を繰り返す.
\begin{enumerate}
  \item クライアントが名簿管理に関する標準入力からの入力をサーバにメッセージとして送信する
  \item サーバがクライアントからのメッセージを受信する
  \item サーバがメッセージに対応した名簿データに関する処理を行い,その結果をクライアントにメッセージとして送信する
  \item クライアントがサーバーから受信したメッセージを標準出力へ出力する
\end{enumerate}
クライアントとサーバは一方が送信処理をする際にもう一方が受信処理
を行う様に通信を同期してやりとりを行う.
処理の終了時にどちらも接続を切断し,
クライアントはプログラムを終了し,サーバは次のクライアントの
接続を待ち受ける状態となる.


プログラムで扱う名簿データは「ID,氏名,誕生日,住所,備考」の項目からなるCSV形式であり,
サーバはこの形式のデータを受信するとこれを新たな名簿データとしてメモリに保存する.
また,'\verb|%|'から始まるメッセージは名簿データに対する様々な処理を実行するためのコマンド入力であり,
サーバからクライアントへのメッセージは主にこのコマンド処理の結果である.
以下,実装したコマンドに対応した処理や機能,及び処理結果の出力について示す.

\paragraph*{\%Q コマンド}
クライアントプログラムを終了し,サーバープログラムはクライアントとの接続を終了して
次のクライアントプログラムの接続を待ち受ける状態となる.

\paragraph*{\%C コマンド}
登録した名簿データ数の表示を行う.
サーバからクライアントへ登録されている名簿データの数と,登録可能な残りのデータ数に
関するメッセージを送信する.

\paragraph*{\%P コマンド}
引数に入力された整数値に応じて登録されている名簿データを表示する.
サーバはクライアントへ該当の名簿データ全ての各項目を表示した文字列を
メッセージとして送信する.
引数の仕様を以下に示す.
\begin{itemize}
  \item $0$の場合:登録順に全件表示
  \item 負の場合:登録データの後ろから順に絶対値の数だけ表示
  \item 絶対値が総データ数以上の場合:登録順に全件表示
\end{itemize}

\paragraph*{\%R コマンド}
サーバ側で引数に指定された名前のファイルを読み込み,CSV形式で記された文字列を
名簿データとして登録を行う.

\paragraph*{\%W コマンド}
サーバ側で引数に指定された名前のファイルへ,登録されている
全ての名簿データをCSV形式で出力する.

\paragraph*{\%S コマンド}
引数で指定された整数値に対応する項目に関して,サーバプログラムのメモリに登録されている
名簿データのソートを行う.

\paragraph*{\%F コマンド}
引数に入力された文字列と完全に一致する項目を持つ名簿データ全ての
表示を行う.\%Pコマンドと同様に,サーバは該当の名簿データの情報を
メッセージとしてクライアントへ送信する.

\paragraph*{\%D コマンド}
引数に入力された整数値に応じて,サーバがメモリに保存している
名簿データを削除する機能を持つ.
引数の仕様を以下に示す.
\begin{itemize}
  \item $0$の場合:全データ削除
  \item 正の場合:先頭から引数の数だけ削除
  \item 負の場合:登録データの後ろから順に引数の絶対値の数だけ表示
  \item 絶対値が総データ数以上の場合:全データ削除
\end{itemize}

\subsection{クライアントプログラムの機能と作成方針} \label{sec:func-client}
クライアントは始めにサーバとのTCPコネクションの確立を行う処理をする.
接続にはサーバが動作する計算機のホスト名と,サーバプログラムのポート番号
が必要であり,これらの情報はプログラム実行時の引数として
入力し,これをそれぞれIPアドレス,ポート番号として接続の処理
に利用するという設計にした.

サーバとの接続が確立した後は,標準入力からの入力された文字列を
サーバへメッセージとして送信し,その後サーバの処理結果を
受信して標準出力へ出力するという処理を繰り返す.
この繰り返しの実現には,TCPの接続処理の後にループの構文を設け,
その中でメッセージの入力と送信,受信と出力の処理を行うという方針を取った.

\%Qコマンドによるプログラムの終了は,
処理の複雑化を避けるため,上記のサーバとのメッセージの送受信・同期の処理から
分岐せず,サーバからのメッセージを受信した後にループから抜ける
という設計にした.

\subsection{サーバプログラムの機能と作成方針}
サーバープログラムは,まず通信相手となるクライアントからの
接続要求を待ち受ける.この機能ではクライアント側が接続をするために,プロセス
のポート番号を設定する必要があり,これをプログラム実行時の
引数として入力された文字列を整数値として変換することで
実装を行った.

クライアントとの接続後は以下の処理を行う.
\begin{enumerate}
  \item クライアントの接続を受け付ける
  \item クライアントからのメッセージを受信する
  \item メッセージに対応した名簿管理の処理を行う
  \item 処理の結果をクライアントに送信,2の処理へ
  \item クライアントとの接続終了後,1に戻る
\end{enumerate}
処理2--4はクライアントプログラムと同様に,クライアントとの
やり取りが終了するまで繰り返されるため,ループ処理を設けて
その中に実装する方針とした.
クライアントから\%Qコマンドが送信された場合は
接続の終了を示すメッセージを送信してクライアントとの接続を終了した後,
プログラムはプロセスを終了せず,次のクライアントプログラムの接続を待ち受ける
ため,処理1--5を無限ループの処理の中に設けることで実現した.
そのため,サーバプログラムの終了はキーボードのCtrl-C入力もしくは
\verb|kill|コマンドで行う.
Ctrl-C入力は本来強制的にプロセスを終了させるが,本プログラムは第\ref{sec:ad3-crtlc}
項に示す様に安全に終了する機能を持つ.


\subsubsection{多重通信受付対応機能 (発展課題2)} \label{sec:ad2-multi}
今回作成したサーバプログラムは複数のクライアントのプロセスから
の接続を同時に処理することが可能な多重通信受け付け機能を持つ.
実現にあたっては,
接続の待ち受けをする処理と
コネクションを確立し名簿管理のデータ通信を行う処理を
分けるという仕組みが有効であると考え,
接続要求の待ち受けを主に行う元のサーバプログラムを親プロセスとして
TCPコネクションの確立の後にそれぞれのクライアントに対応した通信処理を行う
プロセスを複製する方針により実装を行った.
機能の確認のため,サーバは複製したプロセスのID及び
メッセージのやり取りを行ったクライアントとの通信に使用している
ソケットのディスクリプタの値を表示する.
処理の概要については第\ref{sec:prog-server}節に示す.

\subsubsection{サーバログ機能 (発展課題3)} \label{sec:ad3-log}
社会で利用されるサーバプログラムやシステムでは
情報の管理・分析やセキュリティ対策のために接続元の
IPアドレス等の情報,アクセスした時刻やイベントを
ログとして記録することが多い.
そのため,本演習で作成したプログラムにおいても
サーバ側で様々な情報をログとして記録する機能を実装した.

ログに記録する情報は接続元のIPアドレスとポート番号,
実行されたコマンドとメッセージにより登録された名簿データである.
ログファイルの作成は
クライアントとのTCPコネクション確立に行い,
メッセージの送受信のループ処理の直前で
その時点での時刻と接続元のIPアドレスとポート番号をファイルへ書き込む.
クライアントから受信したデータに関しては,実行可能なコマンド
と登録可能な形式の名簿データのみを書き込む.

\subsubsection{クライアントプログラム停止時の一時データ記録と復帰処理 (発展課題3)} \label{sec:ad3-original}
サーバプログラムは,クライアントとのやりとりと通信の終了を\%Qコマンドのメッセージ受信により行う.
しかし,クライアントプログラムがヒューマンエラー等の理由によりCtrl-C入力で
終了した場合は扱っていた名簿データが保存されないまま失われてしまう.
そのため,サーバ側でクライアントがCtrl-C入力で終了したことを検知し,
それを異常終了とみなして一時的にファイルへその時点で保存している名簿データを書き出し,
以降接続するクライアントにその一時データを読み込む選択ができる機能を実装した.

一時的に記録したファイルがある場合は新たに接続したクライアントに
メッセージを送信し,そのデータを読み込むがどうかを選択させ,
選択された場合はファイルの内容を名簿データとして読み込んだ上で一時ファイルを削除する.
選択されなかった場合はそのファイルを削除せず,データも読み込まない.

\subsubsection{Ctrl-C入力時の安全なプログラムの終了処理 (発展課題3)} \label{sec:ad3-crtlc}
サーバプログラムはクライアントの接続要求を待ち受けるために
無限ループの処理を設けており,プロセスの終了には
Ctrl-C入力により強制終了させる方法がある.
しかし,この場合では通信で使用している待ち受け用のソケット,
クライアント用のソケットの削除,\ref{sec:ad3-log}項で示した
ログファイルのファイルディスクリプタの削除など,
本来プロセスを終了する前に行うべき処理が行われない.
そのため,サーバプログラムではCtrl-C入力時にこれらの処理を行い安全に終了し,
正しくファイルやソケットのディスクリプタを削除したメッセージを標準出力へ標準する.
機能を実装した.この機能では,上記の処理に
状態のままになることを防ぐために
Ctrl-C入力時にはプロセスにキーボードからの割り込みを示す
シグナルが送信されるため,機能の実装にはプログラムで\verb|signal| システムコール
を利用して,このシグナル発生時に上記の行うべき処理を行う方針をとった.

%--------------------------------------------------------------------%
\section{プログラムの処理の説明}
サーバ・クライアントのプロセスは第\ref{sec:prog-flow}節
で示した様にTCPコネクションを確立した後に,相互に送信と受信の対応を同期させて
メッセージのやりとりを行う.これら一連の処理の流れについて
\begin{enumerate}
  \item サーバ・クライアントのコネクションの確立
  \item サーバの安全なプロセス終了処理の設定(発展課題3)
  \item クライアントの標準入力からの入力データ取得・サーバへのメッセージ送信
  \item サーバのメッセージ受信
  \item サーバの受信メッセージによる名簿データの処理,クライアントへの処理結果のメッセージ送信
  \item クライアントのメッセージ受信と標準出力への出力
\end{enumerate}
に分けて解説する.

\subsection{サーバ・クライアントのコネクションの確立}
\subsubsection{サーバプログラム} \label{sec:prog-server}
サーバプログラムは,始めにクライアントの接続要求の受け付け処理を行う.
まずソケットの待ち受けの設定を行うために\verb|sockaddr_in|構造体の変数を
宣言し,これに対して
\begin{itemize}
  \item ソケットで利用するアドレスタイプにIPv4(\verb|AF_INET|)を設定する
  \item プログラム実行時の引数で受け取った文字列をポート番号として\verb|atoi|関数により変換
  した整数値を\verb|htons|関数を用いてネットワークバイトオーダに変換して設定する
  \item どのアドレスからの要求を受け付けるために,受け付けるIPアドレスの項目を\verb|INADDR_ANY|として
  \verb|htonl|関数によりネットワークバイトオーダに変換して設定する.
\end{itemize}
という処理を行う.
次に\verb|socket|関数にソケットで利用するアドレスタイプとして第1引数にIPv4を指定する\verb|AF_INET|を,
第2引数に通信形式としてTCPを指定する\verb|SOCK_STREAM|を設定して,ソケットの生成及びその戻り値として
ディスクリプタの値を受け取る.このディスクリプタの値と先述の\verb|sockaddr_in|
構造体の変数を用いて\verb|bind|関数によりソケットに待ち受けるための設定を行う.

待ち受け用のソケットを作成した後,そのディスクリプタを引数として
\verb|listen|関数によりサーバとしてクライアントの接続待ち受けを開始する.
第2引数に設定する一度に通信する上限のクライアント数は$5$と設定した.
次に,クライアントとの接続要求の受付を行うためのループ処理に進む.
この中の処理の始めで待ち受け用のソケットのディスクリプタを引数として\verb|accept|関数で
接続要求を受け取り,クライアントとの通信に使用するソケットを生成する.
この時,\verb|accept|関数の引数に\verb|adddinfo_in|
構造体の変数のポインタを\verb|sockaddr|構造体のポインタにキャストして渡し,
クライアントの情報を取得し,この変数を用いて第\ref{sec:ad3-log}項で示したログ機能により
\verb|open|関数によるクライアント用のログファイルの作成とクライアントのIPアドレスとポート番号,時刻の記入を行う.


クライアントとの通信を行うソケットの生成を行った後,第\ref{sec:ad2-multi}項で示した多重通信受付
のために\verb|fork|関数によりその時点での処理の進行,メモリ状態が同じであるプロセスの複製を行う.
この時の戻り値として受け取ったプロセスIDが0より大きい親プロセスと
プロセスIDが0である複製された子プロセス
で処理を\verb|if|文により分岐させ,それぞれ以下の処理を行う.
\begin{description}
  \item[親プロセス] 接続したクライアントとのメッセージ送受信の処理は行わず,再びループの始めに戻り,\verb|accept|関数で次のクライアントの接続待ち受ける.
  \item[子プロセス] 始めに第\ref{sec:ad3-original}項で示した一時記録された名簿データの読み込み
  を選択する処理を行う.その後,クライアントとのメッセージのやりとりを行うループの処理に入る.
  クライアントとの通信の終了後はクライアントの接続要求を受け付けるループ処理を抜け,クライアントとの通信用ソケットを\verb|close|関数で削除してプロセスを終了する.
  また,この時ログ機能により作成したファイルのディスクリプタも\verb|close|関数により削除する.
\end{description}
これによって
\begin{itemize}
  \item 親プロセスはクライアントの接続要求を受け続ける処理を繰り返す
  \item 子プロセスはクライアントとのメッセージの送受信を行い,通信終了後にプロセスを終了する
\end{itemize}
という処理の流れになり,複数のクライアントのプロセスからの接続要求を同時に処理する
ことができる.
また,サーバプログラムでは,クライアントとのコネクションを確立してプロセスを
複製すると,親プロセスが子プロセスのプロセスIDを標準出力へ表示する.

\subsubsection{クライアントプログラム}
クライアントプログラムは始めに,
接続先のサーバプロセスとのTCPコネクションの確立の処理をする.
サーバに接続するために必要なサーバプロセスの
ホスト名とポート番号はプログラム実行時に引数として文字列データとして
受け取るため,こを引数として
\verb|getaddrinfo|関数によって,ソケットの生成に必要な項目の値が設定された\verb|addrinfo|
構造体のデータを取得する.
このデータを利用して\verb|socket|関数によりサーバとの接続に利用する
ソケットのディスクリプタを得て,この値を引数に\verb|connect|関数により
サーバとの接続を確立する.
コネクションの確立後は\verb|while|文の中で
以降解説するサーバとのやりとりを繰り返し行う.

\subsection{サーバの安全なプロセス終了処理の設定(発展課題3)} \label{sec:prog-ad3-original}
サーバでは,第\ref{sec:ad3-crtlc}項に示したCtrl-C入力時の安全なプロセスの
終了処理を待ち受け用ソケットを作成した後に設けている.
プログラムでは\verb|signal|関数により,Ctrl-C入力時にプロセスに送信される\verb|SIGINT|
シグナルを第1引数として,第2引数にプロセス終了前に行う処理をまとめた関数を渡す.
この関数では,待ち受け用のソケットのディスクリプタとログファイルのディスクリプタの削除を行う.
この時,子プロセスは削除したログファイルのディスクリプタとクライアント通信用ソケットディスクリプタの値を表示し,
親プロセスはクライアントんも接続待ち受けに使用していたソケットのディスクリプタの値を表示する.
\subsection{クライアントの標準入力からのデータ取得・サーバへのメッセージ送信} \label{sec:prog-client-send}
標準入力からの入力データの受け取りとサーバへのメッセージの送信は
同じ\verb|while|文中で行い,以下の流れで一連の処理を行う.

まず標準入力からの
入力データをシステムコール関数の\verb|read|により取得し,サイズが$10$バイトのバッファー(\verb|char|型の配列)
に格納する.そしてこのバッファーと\verb|read|関数の戻り値である取得したバイト数を引数として,
\verb|send|関数によりサーバへメッセージを送信する.
ユーザからのメッセージ入力の終了はEnterキーの押下による改行文字を目印とし,
\verb|read|関数の処理後に取得バイト数を利用して入力文字列の末尾の改行文字(\verb|'\n'|)があるかを確認し,
あればその要素を送信終了を示す目印としてメッセージで利用しないASCIIコードの整数値$3$(\verb|0x03|)に変更し,
\verb|send|関数により送信した後にループ処理から抜ける.
その後,サーバからのメッセージを受信する処理へ移行する.

\subsection{サーバのメッセージ受信}
クライアントからのデータ受信はクライアントの送信処理と同様に\verb|while|文中で
システムコール関数の\verb|read|により取得し,サイズが$10$バイトの受信用バッファー(\verb|char|型の配列)
に格納することを繰り返す.ここで,名簿管理に使用するメッセージは連結した1つの文字列データとして
扱うため,予め宣言した最大入力文字数($1024$)$ + 1$を格納できる\verb|char|型のメッセージ用バッファー
へ読み込んだバッファーの内容をコピーを行う.この処理は以下の流れにより行う.
\begin{enumerate}
  \item \verb|while|文の直前で名簿管理処理用のメッセージ用バッファーの先頭アドレスをポインタ変数に格納する
  \item \verb|while|文の繰り返し処理において,\verb|read|関数の処理後,
  戻り値の取得バイト数だけ\verb|strncpy|関数により受信用バッファーの内容を先述のポインタの位置へコピーする
  \item 上記ポインタ変数を取得バイト数だけ加算する.
\end{enumerate}
これにより,繰り返し受信したデータを連結したメッセージとして保存を行う.
受信処理の終了は,\verb|read|関数の処理後に
受信用バッファー内にクライアントプログラムにより付与された送信終了の目印である
整数値$3$(\verb|0x03|)があるかを確認し,
ある場合はその要素を終端文字(\verb|'\0'|)へ変更し,メッセージ用バッファーへのコピー後
に受信のループ処理を抜ける.
このメッセージ中の送信終了の目印により,クライアントの\verb|send|による送信回数に依らず,
サーバはメッセージの受信を適切に終了を行う.

また,多重通信受付対応の確認のため,メッセージを送信してきたクライアントとの
通信に使用しているソケットのディスクリプタの値を表示する.

\subsection{サーバの受信メッセージによる名簿データの処理・クライアントへのメッセージ送信}
メッセージの受信後,メッセージ用バッファーに格納した文字列を解析して名簿管理に関する処理を行う.
この時,メッセージが実行可能なコマンドまたは登録可能な名簿データである場合は第\ref{sec:ad3-log}
項で示したログファイルに書き込む.
処理の結果,クライアント側にメッセージを送る場合は,
クライアントの送信処理と同様にサイズが$10$バイトの送信用バッファーを用いて,
終端文字も含めたメッセージのバイト数を$10$バイトずつ分割して
必要な回数だけ\verb|send|関数によりクライアントへデータの送信を繰り返す.

この処理の後,
受信メッセージが\%Qコマンドで
ない場合はサーバのメッセージ送信終了の目印として,整数値3(\verb|0x03|)を代入した
1バイトのデータを\verb|send|関数により送信し,次のクライアントプログラムからのメッセージを受信する処理へ
移行する.\%Qコマンドであった場合は,
このデータをクライアントへ接続終了を示す目印として,メッセージ終了の目印と同様に
通常のメッセージで使用されないASCIIコードの整数値$4$(\verb|0x04|)を代入して送信し,
クライアントとのメッセージのやりとりを行うループ処理を抜け,クライアントとの通信用ソケットを\verb|close|関数
により削除する.

以上の様に,コマンドにより名簿管理に関するメッセージの送信をするかどうかに
関わらず,必ずクライアント側へサーバの処理が終了したことを表す目印を
メッセージと送信を行う.これによりクライアントの受信処理を終了させ,送受信の同期を行う.

\subsection{クライアントのサーバからのメッセージ受信・標準出力への出力}
サーバからのメッセージの受信と標準出力への出力は第\ref{sec:prog-client-send}項
で解説した処理と同様に,\verb|while|文中で行い,
以下の流れで処理を行う.

まずサーバからの送信されたデータを\verb|recv|関数によりサイズが$10$バイトの受信用バッファー
に格納し,
この受信用バッファーのメッセージをシステムコール関数の\verb|write|関数により,標準出力へ
\verb|recv|関数の戻り値である取得したバイト数だけ出力する.
メッセージ出力処理の後はサーバのクライアントメッセージの受信処理と同様に
バッファー内にサーバが施したメッセージの終了に関する目印に該当する要素があるかを確認し,あれば以下の処理を行う.
\begin{description}
  \item[整数値$3$(\texttt{0x03})] メッセージの送信終了を示し,
  受信処理のループを抜け,再びユーザからの入力とサーバへのメッセージ送信
  を行う処理へ移行する.
  \item[整数値$4$(\texttt{0x04})] \%Qコマンド入力判定による通信の接続終了を示し,
  サーバとのメッセージ送受信処理のループを抜け,ソケットを\verb|close|関数により削除してプロセスを終了する.
\end{description}
また,\verb|recv|関数による受信データの取得後は,\verb|recv|関数の戻り値が$0$であり,接続相手のサーバが通信していたソケットを閉じた場合は
上記の\%Qコマンド入力判定時の接続終了と
同じ処理を行う.

%--------------------------------------------------------------------%
\section{プログラムの使用方法と使用例及び動作結果}
本実験で作成したサーバ・クライアントのプログラムの,
実際に使用する方法,使用例及びそれらの動作結果について
以下の操作に関して示す.

\begin{itemize}
  \item サーバ・クライアントプログラムの起動と通信の接続
  \item 名簿管理に関するコマンドの実行
  \item 他学生のサーバプログラムとの通信
  \item サーバの多重通信受付対応
  \item Ctrl-C入力でのサーバプロセスの安全な終了
  \item Ctrl-C入力でのクライアント停止時の一時データ記録と復帰処理
\end{itemize}

\subsection{サーバ・クライアントプログラムの実行と通信の接続}
サーバプログラムでは,通信相手からの接続要求を受け付けるため,
プロセスのポート番号を設定する必要がある.これはプログラム実行時
に入力された引数を利用する.
\begin{itemize}
  \item 実行ファイル ポート番号
\end{itemize}

例としてポート番号を$60000$として
待ち受ける際の使用例を以下に示す.サーバプログラムの実行ファイル名は
\verb|meibo_server|としている.

\begin{Verbatim}[numbers=none, numbersep=6pt, frame=single,
                    fontsize=\small, baselinestretch=0.8]
$ ./meibo_server 60000
\end{Verbatim}
これにより,引数で与えたポート番号でクライアントの接続を受け付ける状態となる.

クライアントプログラムでは,サーバとの接続のために,サーバの
ホスト名とポート番号を設定する必要がある.これもサーバプログラム
と同様にプログラム実行時の引数として与え,第1引数がホスト名(IPアドレスまたはドメイン名),
第2引数がポート番号となっている.
\begin{itemize}
  \item 実行ファイル ホスト名 ポート番号
\end{itemize}
同一計算機上で上記の例で示したサーバが動作している場合のクライアントの使用例を以下に示す.
クライアントの実行ファイル名は\verb|meibo_client|としている.

\begin{Verbatim}[numbers=none, numbersep=6pt, frame=single,
  fontsize=\small, baselinestretch=0.8]
$ ./meibo_client localhost 60000
[ Server Info ] IP Address: [ 127.0.0.1 ] - Port: [ 60000 ]
\end{Verbatim}
プログラム実行後,プロセスが終了しない状態であればサーバとの接続
が完了し,名簿管理に関する処理が行えるようになる.

\subsection{名簿管理に関するコマンドの実行}
以下,サーバと接続後のクライアントプログラムでの各コマンドを実行した際の
動作例を示す.サーバは前節の例と同様に同一計算機上でポート番号60000
で接続要求を待ち受けているものとする.

\subsubsection{\%Q コマンド}
以下,サーバへの接続直後に\%Qコマンドを実行した際の動作例を示す.

\begin{Verbatim}[numbers=none, numbersep=6pt, frame=single,
  fontsize=\small, baselinestretch=0.8]
$ ./meibo_client localhost 60000
[ Server Info ] IP Address: [ 127.0.0.1 ] - Port: [ 60000 ]
%Q
See you!
\end{Verbatim}
コマンド実行後.\verb|See you!|のメッセージが表示され,プロセスが終了する.

\subsubsection{\%C,R,P コマンド}
以下,\%C,R,Pコマンドを使用した際の動作例を示す.
機能の動作を確認する上で名簿データが必要であったため,そのデータ
として実験で配布された\verb|sample.csv|ファイルを使用している.
このファイルには2886データが記録されている.
コマンド操作は以下の順で行った.
\begin{enumerate}
  \item \%C
  \item \%P (全データ表示)
  \item \%R (\verb|sample.csv|の読み込み)
  \item \%C
  \item \%P (先頭から3データ)
\end{enumerate}
\begin{Verbatim}[numbers=none, numbersep=6pt, frame=single,
  fontsize=\small, baselinestretch=0.8]
%C
0 profile(s)
Enable to add 10000 data(s)
%P 0
%R sample.csv
%C
2886 profile(s)
Enable to add 7114 data(s)
%P 3
Id    : 5100046
Name  : The Bridge
Birth : 1845-11-02
Addr. : 14 Seafield Road Longman Inverness
Comm. : SEN Unit 2.0 Open

Id    : 5100127
Name  : Bower Primary School
Birth : 1908-01-19
Addr. : Bowermadden Bower Caithness
Comm. : 01955 641225 Primary 25 2.6 Open

Id    : 5100224
Name  : Canisbay Primary School
Birth : 1928-07-05
Addr. : Canisbay Wick
Comm. : 01955 611337 Primary 56 3.5 Open  
\end{Verbatim}
\%Rコマンドによるデータの読み込みにより,\%Cコマンドで表示される
データ数がファイルに登録されていた数になり,\%Pコマンドでも
指定した数だけ名簿データの表示がされた.

\subsubsection{\%W コマンド}
\%Wコマンドによる登録データのファイルへの書き出しの
動作例を示す.
動作確認のため,\%Wコマンドも含めて以下の操作を順に行った.
\begin{enumerate}
  \item 適当な名簿データを3つ登録する
  \item \%Cコマンドでデータ数を確認する
  \item \%Wコマンドで\verb|copy.csv|という名前のファイルにデータを書き出す.
\end{enumerate}

\begin{Verbatim}[numbers=none, numbersep=6pt, frame=single,
  fontsize=\small, baselinestretch=0.8]
100,takahashi,2001-1-1,tokyo,no comment.
200,suzuki,2002-2-2,osaka,no comment.
300,honda,2003-3-3,kyoto,no comment.
%C
3 profile(s)
Enable to add 9997 data(s)
%W copy.csv  
\end{Verbatim}
実行後の\verb|copy.csv|の内容を以下に示す.コマンド実行時点で登録していた
名簿データが記録されている.
\begin{Verbatim}[numbers=left, numbersep=6pt, frame=single, xleftmargin=10mm,
  fontsize=\small, baselinestretch=0.8]
100,takahashi,2001-1-1,tokyo,no comment.
200,suzuki,2002-2-2,osaka,no comment.
300,honda,2003-3-3,kyoto,no comment.
\end{Verbatim}

\subsubsection{\%S コマンド}
前項で登録した3つの名簿データがある状態で\%Sコマンドにより
データをソートした際の実行結果を示す.
操作は以下の順で行った.
\begin{enumerate}
  \item \%Pコマンドで全データを表示する
  \item \%Sコマンドで名前(第2項目)についてソートする
  \item \%Pコマンドで全データを表示する
\end{enumerate}

\begin{Verbatim}[numbers=none, numbersep=6pt, frame=single,
  fontsize=\small, baselinestretch=0.8]
%P 0
Id    : 100
Name  : takahashi
Birth : 2001-01-01
Addr. : tokyo
Comm. : no comment.

Id    : 200
Name  : suzuki
Birth : 2002-02-02
Addr. : osaka
Comm. : no comment.

Id    : 300
Name  : honda
Birth : 2003-03-03
Addr. : kyoto
Comm. : no comment.

%S 2
%P 0
Id    : 300
Name  : honda
Birth : 2003-03-03
Addr. : kyoto
Comm. : no comment.

Id    : 200
Name  : suzuki
Birth : 2002-02-02
Addr. : osaka
Comm. : no comment.

Id    : 100
Name  : takahashi
Birth : 2001-01-01
Addr. : tokyo
Comm. : no comment.  
\end{Verbatim}
名前の項目でアルファベット順にソートされたことが確認できる.

\subsubsection{\%F コマンド}
\%W,\%Sコマンドの動作例で使用した3つの名簿データが登録された状態で\%Fコマンドにより
引数で指定した文字列と一致する項目でを持つ名簿データの検索を行った実行結果を示す.
操作は以下の順で行った.
\begin{enumerate}
  \item 「\verb|100|」に一致する項目を持つデータを検索する
  \item 「\verb|no comment.|」に一致する項目を持つデータを検索する
\end{enumerate}

\begin{Verbatim}[numbers=none, numbersep=6pt, frame=single,
  fontsize=\small, baselinestretch=0.8]
%F 100
[No.3]
Id    : 100
Name  : takahashi
Birth : 2001-01-01
Addr. : tokyo
Comm. : no comment.

%F no comment.
[No.1]
Id    : 300
Name  : honda
Birth : 2003-03-03
Addr. : kyoto
Comm. : no comment.

[No.2]
Id    : 200
Name  : suzuki
Birth : 2002-02-02
Addr. : osaka
Comm. : no comment.

[No.3]
Id    : 100
Name  : takahashi
Birth : 2001-01-01
Addr. : tokyo
Comm. : no comment.  
\end{Verbatim}

\subsubsection{\%D コマンド}
\%Dコマンドによる登録データ削除機能の
動作例を示す.
動作確認のため,他のコマンドも含めて以下の操作を順に行った.
\begin{enumerate}
  \item \%Rコマンドで\verb|sample.csv|ファイルのデータを読み込む(2886データ)
  \item \%Cコマンドで登録データ数を確認する
  \item \%Dコマンドで先頭から2880データを削除
  \item \%Cコマンドで登録データ数を確認する
\end{enumerate}

\begin{Verbatim}[numbers=none, numbersep=6pt, frame=single,
  fontsize=\small, baselinestretch=0.8]
%R sample.csv
%C
2886 profile(s)
Enable to add 7114 data(s)
%D 2880
%C
6 profile(s)
Enable to add 9994 data(s)  
\end{Verbatim}

\subsection{他学生のサーバプログラムとの通信 (発展課題1)}
作成したクライアントプログラムを他学生の作成したサーバプログラム
に接続する際の使用例を以下に示す.

サーバプログラムは学生番号:09B22509,氏名:垣本桃弥さん
が作成したものを使用した.

動作確認に関する操作は以下の順で行った.
\begin{enumerate}
  \item 相手方のサーバプロセスが動作している計算機のIPアドレス,ポート番号を引数としてクライアントプログラムを実行する
  \item \%Cコマンドでデータ数を確認する
  \item \%Rコマンドで\verb|sample.csv|ファイルのデータを読み込む(ファイルはサーバ側のものが対象)
  \item \%Cコマンドでデータ数を確認する
  \item \%Pコマンドで先頭から3データを表示する
  \item \%Qコマンドでサーバとの通信を終了,クライアントプログラムを終了する
\end{enumerate}

\begin{Verbatim}[numbers=none, numbersep=6pt, frame=single,
  fontsize=\small, baselinestretch=0.8]
$ ./meibo_client 61003
%C
0 profile(s)
%R sample.csv
%C
2886 profile(s)
%P 3
Id    : 5100046
Name  : The Bridge
Birth : 1845-11-02
Addr. : 14 Seafield Road Longman Inverness
Comm. : SEN Unit 2.0 Open

Id    : 5100127
Name  : Bower Primary School
Birth : 1908-01-19
Addr. : Bowermadden Bower Caithness
Comm. : 01955 641225 Primary 25 2.6 Open

Id    : 5100224
Name  : Canisbay Primary School
Birth : 1928-07-05
Addr. : Canisbay Wick
Comm. : 01955 611337 Primary 56 3.5 Open

%Q
$
\end{Verbatim}

\subsection{サーバログ機能 (発展課題3)}
サーバログ機能は,接続したクライアントそれぞれのログファイルを作成する.
ログファイルには
\begin{itemize}
  \item IPアドレス
  \item ポート番号
  \item 接続した時刻
  \item 実行したコマンド・入力により登録した名簿データ
\end{itemize}
を記入する.
ログファイル名は接続時の時刻に基づき,
\verb|年-月-日_時-分-秒.log|となる.
以下,クライアントが接続し,以下の操作を行った際のサーバプロセスの出力と
作成されたログファイルの記録について示す.
\begin{enumerate}
  \item \%Cコマンドの実行
  \item 無効なコマンドの入力
  \item 新たな名簿データの入力
  \item \%Pコマンドで全データを表示
  \item \%Rコマンドによる\verb|sample.csv|の読み込み
  \item \%Qコマンドによる接続の終了
\end{enumerate}

\paragraph*{サーバプロセスの出力}
\begin{Verbatim}[numbers=none, numbersep=6pt, frame=single,
  fontsize=\small, baselinestretch=0.8]
$ ./meibo_server 60000
log file: log/2025-01-19_16-32-49.log
<connect> 	IP [127.0.0.1] PORT [54344] SOCKET: [4]
make process pid: 36692
msg: [%C] #sockfd: 4
msg: [%A] #sockfd: 4
msg: [100,takahashi,2001-1-1,toyko,no comment.] #sockfd: 4
msg: [%P 0] #sockfd: 4
msg: [%R sample.csv] #sockfd: 4
msg: [%Q] #sockfd: 4
<disconnect> 	IP [127.0.0.1] PORT [54344] SOCKET: [4]
\end{Verbatim}

\paragraph*{ログファイルの内容}
\begin{Verbatim}[numbers=left, numbersep=6pt, frame=single, xleftmargin=10mm,
  fontsize=\small, baselinestretch=0.8]
[TIME] 2025/01/19 16:32:49
[IP]   127.0.0.1
[PORT] 54344
%C
100,takahashi,2001-1-1,toyko,no comment.
%P 0
%R sample.csv
%Q
\end{Verbatim}
仕様の通り,有効なコマンドと新たに登録した名簿データが記録されていることが確認できた.


\subsection{Ctrl-C入力でのサーバプロセスの安全な終了 (発展課題3)}
サーバプロセスが実行時にCtrl-C入力された際,サーバは通信に使用していたソケットと
ログファイルのディスクリプタを\verb|close|関数で削除を行い,その値を表示する.

以下,1つのクライアントプログラムが接続されている状態でサーバプロセスをCtrl-Cで終了させた際の
サーバ及びクライアントの出力結果を示す.クライアントがサーバプログラムの終了をメッセージの受信により
判別するため,一度何かしらの入力を行いメッセージ送信処理を終える必要があるため,コマンドを入力している.

\paragraph*{サーバプロセスの出力}
\begin{Verbatim}[numbers=none, numbersep=6pt, frame=single,
  fontsize=\small, baselinestretch=0.8]
$ ./meibo_server 60000
make process pid: 45549
<connect> 	IP [127.0.0.1] PORT [58376] SOCKET: [4]
^C[EXIT] close log fd: 5
[EXIT] close listen socket fd: 3
[EXIT] close client socket fd: 4
\end{Verbatim}

\paragraph*{クライアントプロセスの出力}
\begin{Verbatim}[numbers=none, numbersep=6pt, frame=single,
  fontsize=\small, baselinestretch=0.8]
$ ./meibo_client localhost 60000
[ Server Info ] IP Address: [ 127.0.0.1 ] - Port: [ 60000 ]
%C
[ERROR] Server is terminated.
$
\end{Verbatim}
サーバ側では
\begin{itemize}
  \item クライアントの接続待ち受け用のソケットのディスクリプタ
  \item クライアントとの通信用ソケットのディスクリプタ
  \item クライアントのログファイルのディスクリプタ
\end{itemize}
が閉じられた出力が,クライアント側では
サーバ側のプロセスが終了したことを示すメッセージが出力されて
プロセスが終了することが確認できた.

\subsection{Ctrl-C入力でのクライアント停止時の一時データ記録と復帰処理 (発展課題3)}
第\ref{sec:ad3-original}項で示した,クライアントプログラム停止時の
一時的なデータの記録と,そのデータの以降のクライアント接続時の復元について,
サーバに対して1つのクライアントプログラムで接続し,
以下に示す操作で動作を確認した.この動作確認時点では一時記録ファイルは
無い状態である.
\begin{enumerate}
  \item \%Rコマンドで\verb|sample.csv|の2886個のデータを読み込む
  \item \%Dコマンドで2880個のデータを削除,6個のデータを残す
  \item \%Cコマンドでデータ数を確認する
  \item Ctrl-C入力でクライアントのプロセスを終了させる
  \item 再度クライアントプログラムを実行
  \item 直前のデータを復元し読み込むことを選択する
  \item \%Cコマンドでデータ数を確認する
\end{enumerate}

操作を行った際のクライアントプログラムの出力を示す.
\begin{Verbatim}[numbers=none, numbersep=6pt, frame=single,
  fontsize=\small, baselinestretch=0.8]
$ ./meibo_client localhost 60000
[ Server Info ] IP Address: [ 127.0.0.1 ] - Port: [ 60000 ]
%R sample.csv
%D 2880
%C
6 profile(s)
Enable to add 9994 data(s)
^C
$ ./meibo_client localhost 60000
[ Server Info ] IP Address: [ 127.0.0.1 ] - Port: [ 60000 ]
Return to the state before error exit ? [y/n] y
%C
6 profile(s)
Enable to add 9994 data(s)
\end{Verbatim}
仕様の通り,クライアントがCtrl-Cで終了した際に,その時点の名簿データが記録され,
以降接続した際に復元するかどうかの選択処理が行われ,データが正しく読み込まれたことが確認できる.
%--------------------------------------------------------------------%
\section{プログラムの作成過程に関する考察}
今回のプログラムにおいて重要な機能であるメッセージの送受信とその同期処理の
実装にあたり,サーバ・クライアントプログラムのどちらも
ユーザからの入力や処理結果によって長さが大きいメッセージを扱う
ことを考慮して,バッファーサイズや処理回数の修正が発生すると考えられる
必要な送受信回数を共有するという方法ではなく,メッセージに処理の終了を
示す目印を加える実装にしたことで,それぞれのプログラムでの送受信に関する処理
の構成が共通化でき,結果として修正や変更が容易になった.
加えて,この送受信処理を関数を利用しまとまった構造にすることで,
基になっている名簿管理プログラムのコードの変更を少なくし,
基本的な仕様が実装しやすく,機能を実装を進めていく上でバグの特定が行い易くなる
よう工夫を行った.

また,送受信終了の処理も必ず目印となるメッセージを受け取り,
その目印により処理を進めるという仕組みにしてプログラム全体を構成したことで,
\%Qコマンドによるプロセスの終了とメッセージの
受信終了の処理や拡張機能の実装と修正が容易になった.

以上の様に通信処理やそれに関するプログラムの流れに自分なりのルールを決めて作業を進めることで
実装と拡張の負担は下がったが,
それ以外の,特に元の名簿管理プログラムを変更しないことや
その通信関連の部分の処理をより簡単な関数の形にすることを意識しすぎた
結果,通信に関わる関数などでプログラム全体で共通で使用するソケットなどの
変数の多くをグローバル変数に変更してしまい,実装が進みコードが増えるに伴って
それに関わる関数の処理や変数名の扱いに関する修正が難しくなってしまったことに
苦労した.
そのため,今回の様に予め実装箇所が多いプログラムを基にした拡張や改良
については,既存のコードの修正をしないことを意識し過ぎることをせず,
実装したい機能の構成をある程度固めた上で,それに合わせて適度に
修正を加えることが必要であると考えられる.
また,今回の通信に使用するソケットディスクリプタやメッセージのバッファーなどの
プログラム全体的で必要となる変数を扱う関数の実装も,グローバル変数の増加
や複雑化を避けるためにも,これら変数を関連させた構造体として扱い関数やプログラムを考えることも
上記の苦労点の1つの解決のアプローチと言える.

また,通信関連の処理では特にデータの受信と
そのデータに対する処理によるエラーが多く発生したため,
\verb|recv|関数の戻り値を用いたエラーのチェックと
本来文字として出力されないメッセージの終了に使用している目印を走査して
表示・可視化させる処理によるデータ送受信の確認を行い,
加えて受信に使用したデータのバッファーを
そのまま次の処理に利用するのではなく,一度別のバッファーにコピーして
それ以降の名簿管理や通信処理の後のデータを比較・表示できるようにして,
\verb|Segmentation Fault|のようなメッセージ操作に関連したエラーが
どこで発生したものなのかを\verb|printf|による出力やgdbによるデバッグ
で特定しやすいよう工夫を施した.

\subsection{他学生のサーバプログラムとの通信におけるメッセージ仕様の工夫}
今回使用させた頂いた他学生のサーバプログラムは,自身で作成した
サーバプログラムと同様に繰り返し処理の中でデータの送受信処理を繰り返し,
メッセージの最後に終了を示す目印を加えることで,メッセージの送受信と
その同期を行う仕様であった.その目印はNULL文字(\verb|'\0'|)であり,
自身のクライアントプログラムでは他の値を利用していた箇所を
このNULL文字に変更することで正しくメッセージのやり取りができるように
目印の修正を行った.また,\%Qコマンドによるクライアントプログラムの終了について,
自身の仕様ではサーバからのメッセージの末尾の値に応じて終了するかどうかを
分岐させていたが,相手方のサーバプログラムではNULL文字のみのデータをサーバに
送信する仕様であったため,ユーザからの入力メッセージで\%Qコマンドである場合に
クライアント側のプロセスを終了できるように処理を加えた.

\subsection{サーバの多重通信受付対応の機能の確認 (発展課題2)}
今回実装するしたサーバの多重通信受付対応の機能は,クライアントとのコネクションの
確立後,\verb|fork|関数によるプロセスの複製を行う.
そのため,サーバプログラムの実行後,
複数のターミナルからクライアントプログラムを
起動し,サーバへ接続した際のサーバ側の複製したプロセスIDやソケットに関する出力と
\verb|ps|コマンドによる実行プロセスの情報により確認を行った.
\verb|ps|コマンドでは,パイプで\verb|grep|コマンドを繋ぎ,
名簿管理プログラムのサーバ・クライアントのプロセスを調べ
,サーバは接続要求待ち受りを行っている親プロセスを除いて
サーバプロセスの数がクライアントプロセスの数と同じになるか,
そしてサーバプログラムで出力される複製時のプロセスIDと一致しているかどうかで
多重通信の受付ができているかをテストした.

%--------------------------------------------------------------------%
\section{得られた結果に関する考察}
今回,クライアントプログラムにおいて,標準入力からの
入力データの取得とメッセージの標準出力への出力は
文字列操作に関する標準ライブラリ関数でなく,\verb|open|関数,
\verb|write|関数をそれぞれ利用した.
これにより文字列の終端文字もデータとして扱われ,入力データを
まとまった文字列という単位で処理が複雑になり,
元の名簿管理プログラムで動作確認で行えていたファイルのリダイレクションによる
コマンドの処理ができなくなっていた.そのため,動作確認や
実際にユーザが使用する際の利便性も考えて,
これらシステムコール関数で取得したデータ中で文字列の区切れとなる
改行文字や終端文字を起点として,データを文字列単位で分割して配列やリストで
まとめて管理できる処理を可能にすることで,より名簿管理に関する操作が
改善され,さらなる機能の拡張に繋がると考えられる.
%--------------------------------------------------------------------%

% Verbatim environment
% プリアンブルで \usepackage{fancyvrb} が必要.
%   - numbers           行番号を表示.left なら左に表示.
%   - xleftmargin       枠の左の余白.行番号表示用に余白を与えたい.
%   - numbersep         行番号と枠の間隙 (gap).デフォルトは 12 pt.
%   - fontsize          フォントサイズ指定
%   - baselinestretch   行間の大きさを比率で指定.デフォルトは 1.0.
% \begin{Verbatim}[numbers=left, xleftmargin=10mm, numbersep=6pt,
%                     fontsize=\small, baselinestretch=0.8]
% #include <stdio.h>

% int main()
% {
%     char s[4] = {'A', 'B', 'C', '\0'};

%     printf("s = %s\n", s);

%     return 0;
% }
% \end{Verbatim}

%--------------------------------------------------------------------%
% 参考文献
%   以下は,書き方の例である.実際に,参考にした書籍等を見て書くこと.
%   本文で引用する際は,\cite{book:algodata}などとすればよい.
% \begin{thebibliography}{99}
%   \bibitem{book:algodata} 平田富雄,アルゴリズムとデータ構造,森北出版,1990.
%   \bibitem{book:label2} 著者名,書名,出版社,発行年.
%   \bibitem{www:label3} WWWページタイトル,URL,アクセス日.
% \end{thebibliography}

%--------------------------------------------------------------------%
%% 本文はここより上に書く(\begin{document}~\end{document}が本文である)
\end{document}
