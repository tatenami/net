% ファイル先頭から\begin{document}までの内容(プレアンブル)については,
% 基本的に { } の中を書き換えるだけでよい.
\documentclass[autodetect-engine,dvi=dvipdfmx,ja=standard,
               a4j,11pt]{bxjsarticle}

%%======== プレアンブル ============================================%%
% 用紙設定:指示があれば,適切な余白に設定しなおす
\RequirePackage{geometry}
\geometry{reset,paperwidth=210truemm,paperheight=297truemm}
\geometry{hmargin=25truemm,top=20truemm,bottom=25truemm,footskip=10truemm,headheight=0mm}
%\geometry{showframe} % 本文の"枠"を確認したければ,コメントアウト

% 設定:図の挿入
% http://www.edu.cs.okayama-u.ac.jp/info/tool_guide/tex.html#graphicx
\usepackage{graphicx}

% 設定:ソースコードの挿入
% http://www.edu.cs.okayama-u.ac.jp/info/tool_guide/tex.html#fancyvrb
\usepackage{fancyvrb}
\renewcommand{\theFancyVerbLine}{\texttt{\footnotesize{\arabic{FancyVerbLine}:}}}

%%======== レポートタイトル等 ======================================%%
% ToDo: 提出要領に従って,適切なタイトル・サブタイトルを設定する
\title{情報工学実験C ネットワーク実験課題レポート \\
       |\large{クライアントサーバーモデルで動作する名簿管理プログラムの作成}|}

% ToDo: 自分自身の氏名と学生番号に書き換える
\author{氏名: 寺岡 久騎 (TERAOKA, Hisaki) \\
        学生番号: 09B22433}

% ToDo: レポート課題等の指示に従って適切に書き換える
\date{出題日: 2024年12月13日 \\
      提出日: 2025年 1月21日 \\
      締切日: 2025年 1月21日 \\}  % 注:最後の\\は不要に見えるが必要.


%%======== 本文 ====================================================%%
\begin{document}
\maketitle
% 目次つきの表紙ページにする場合はコメントを外す
%{\footnotesize \tableofcontents \newpage}

%% 本文は以下に書く.課題に応じて適切な章立てを構成すること.
%% 章=\section,節=\subsection,項=\subsubsection である.

%--------------------------------------------------------------------%
% \section{概要} \label{sec:abstract}


%--------------------------------------------------------------------%
\section{プログラムの処理の概要および作成方針}
本実験では,ネットワーク通信により動作するクライアントサーバーモデルの名簿管理プログラムの作成を行った.
プログラムはサーバー・クライアントの2つからなり,TCPによる相互のデータ送受信で動作する.

サーバはプログラム実行時の引数で指定されたポート番号で接続を待機し,クライアントはプログラム実行時に引数で指定されたホスト名とポート番号
を持つサーバに接続し,TCPコネクションを確立することで名簿管理の処理が開始し,以降はシステムコール関数を用いたソケット通信により
以下の一連の処理を繰り返す.
\begin{enumerate}
  \item クライアントが名簿管理に関する入力をサーバにメッセージとして送信する
  \item サーバがクライアントからのメッセージを受信する
  \item サーバがメッセージに対応した処理を行い,その結果をクライアントにメッセージとして送信する
  \item クライアントがサーバーからメッセージを出力する
\end{enumerate}
処理の終了時にどちらも接続を終了する.

プログラムで扱う名簿データは「ID,氏名,誕生日,住所,備考」の項目からなるCSV形式であり,
サーバはこの形式のデータを受信するとこれを新たな名簿データとしてメモリに保存する.
また,\verb|%|から始まるメッセージは名簿データに対する様々な処理を実行するためのコマンド入力であり,
サーバからクライアントへのメッセージは主にこのコマンド処理の結果である.
以下,実装したコマンドに対応した処理や機能,及び処理結果の出力について示す.

\paragraph*{\%Q コマンド}
クライアントプログラムを終了し,サーバープログラムはクライアントとの接続を終了して
次のクライアントプログラムの接続を待ち受ける状態となる.

\paragraph*{\%C コマンド}
登録した名簿データ数の表示を行う.
サーバからクライアントへ登録されている名簿データの数と,登録可能な残りのデータ数に
関するメッセージを送信する.

\paragraph*{\%P コマンド}
引数に登録されている名簿データを表示する.
サーバはクライアントへ該当の名簿データ全ての各項目を表示した文字列を
メッセージとして送信する.
引数の仕様を以下に示す.
\begin{itemize}
  \item 引数は整数値
  \item $0$の場合:登録順に全件表示
  \item 負の場合:登録データの後ろから順に絶対値の数だけ表示
  \item 絶対値が総データ数以上の場合:登録順に全件表示
\end{itemize}


\subsection{サーバプログラムの機能と作成方針}
サーバープログラムは名簿管理の処理を担い,
の名簿データ及びこれらに対する処理を行うコマンド入力を

\subsubsection{多重通信受付機能 (発展課題2)}

\subsubsection{サーバログ機能 (発展課題3)}

\subsubsection{Ctrl-C入力時の安全なプログラムの終了処理 (発展課題3)}

\subsubsection{プログラム停止時のデータ記録と復帰処理 (発展課題3)}

\subsection{クライアントプログラムの機能と作成方針}

%--------------------------------------------------------------------%
\section{プログラムの処理の説明}

\subsection{サーバプログラム}

\subsection{クライアントプログラム}

%--------------------------------------------------------------------%
\section{プログラムの使用方法と使用例}

\subsection{他学生のサーバプログラムとの通信}

\subsection{多重通信受付機能の使用}

%--------------------------------------------------------------------%
\section{プログラムの作成過程に関する考察}


%--------------------------------------------------------------------%
\section{得られた結果に関する考察}


%--------------------------------------------------------------------%

% Verbatim environment
% プリアンブルで \usepackage{fancyvrb} が必要.
%   - numbers           行番号を表示.left なら左に表示.
%   - xleftmargin       枠の左の余白.行番号表示用に余白を与えたい.
%   - numbersep         行番号と枠の間隙 (gap).デフォルトは 12 pt.
%   - fontsize          フォントサイズ指定
%   - baselinestretch   行間の大きさを比率で指定.デフォルトは 1.0.
% \begin{Verbatim}[numbers=left, xleftmargin=10mm, numbersep=6pt,
%                     fontsize=\small, baselinestretch=0.8]
% #include <stdio.h>

% int main()
% {
%     char s[4] = {'A', 'B', 'C', '\0'};

%     printf("s = %s\n", s);

%     return 0;
% }
% \end{Verbatim}

%--------------------------------------------------------------------%
% 参考文献
%   以下は,書き方の例である.実際に,参考にした書籍等を見て書くこと.
%   本文で引用する際は,\cite{book:algodata}などとすればよい.
% \begin{thebibliography}{99}
%   \bibitem{book:algodata} 平田富雄,アルゴリズムとデータ構造,森北出版,1990.
%   \bibitem{book:label2} 著者名,書名,出版社,発行年.
%   \bibitem{www:label3} WWWページタイトル,URL,アクセス日.
% \end{thebibliography}

%--------------------------------------------------------------------%
%% 本文はここより上に書く(\begin{document}~\end{document}が本文である)
\end{document}
